\chapter{Geometry}
\kactlimport{AngleAreaOrientationSortRotationPerpendicular.cpp}
\kactlimport{ConvexHull.cpp}
\kactlimport{Diameter.cpp}
\kactlimport{DynamicUpperEnvelope.cpp}
\kactlimport{FindSegmentsIntersection.cpp}
\kactlimport{LiChaoTree.cpp}
\kactlimport{PointClassification.cpp}
\kactlimport{PointInPolygon.cpp}
\kactlimport{PointToSegmentDistance.cpp}
\kactlimport{SegmentsIntersection.cpp}
